\documentclass{beamer}
\usepackage{graphicx} % Required for inserting images


\title{\Large\textbf{La vegetazione di Tarvisio}}
\author{Studentessa: Predan Sorsha Rosanna }
\class\centering{Corso di Telerilevamento Geo-Ecologico}
%\date{}

% https://mpetroff.net/files/beamer-theme-matrix/
\usetheme{Rochester}
\usecolortheme{wolverine}

\begin{document}

\maketitle

\AtBeginSection[] 


\section{Obiettivo Studio}

\begin{frame}
{\Large\textbf{Obiettivo Studio}}
 Lo studio vuole verificare i cambiamenti e/o perdita di vegetazione nell zona di Tarvisio, territorio situato in Friuli Venezia Giulia, spesso colpito da patologie vegetali quali la processionaria del pino e il bostrico del castagno. Sono stati presi in considerazione due anni nel periodo che si estende dal 01 febbraio al 30 giugno (2018 - 2022). Le immagini scelte sono state prese dal sito "Copernicus Browser" utilizzando Instrument Sentinel-2 MSI: Multispectral Instrument.
\end{frame}

\begin{frame}
{\Large\textbf{Indice}}
\begin{columns}
    \column{0.5\textwidth}
 \begin{itemize}
     \item Importazione Immagini
     \item Visualizzazione della vegetazione
     \item Calcolo dell'Indice di variabilità
     \item Classificazione delle immagini
     \item Ggplot e Dataframe
     \item Conclusioni
 \end{itemize}
\column{0.5\textwidth}
    \includegraphics[width=.9\linewidth]{PrimaI.png}
\end{columns}
\end{frame}

\section{Importazione Immagini}

\begin{frame}
{\Large\textbf{Importazione Immagini}}
Le immagini selezionate coprono la zona di Malborghetto, Camporosso in Valcanale e Tarvisio, aree montuose maggiormente colpite da stress ambientali.
\vspace{+1cm}
Exporting data
\ change direction
Windowds users: C://comp/Downloads
setwd("C://nome/Downloads")
setwd("C:/Users/user/Desktop/Telerilevamento/Immagini")
\end{frame}

\section{Visualizzazione della vegetazione}

\begin{frame}
{\Large\textbf{Visualizzazione della vegetazione in funzione del NIR}}

Usando la funzione "im.plotRGB()" del pacchetto "imageRy", vado a sostituire il nir con le altre bande rgb per osservare come varia la visualizzazione della vegetazione presente nel 2018 e nel 2022.
  \begin{itemize}
     \item Vegetazione rossa: Nir , Green, Blue 
     \item Vegetazione blu: Red, Green, Nir
     \item Vegetazione verde: Red, Nir, Blue
  \end{itemize}
    \centering
    \includegraphics[width=.3\linewidth]{imageRed.png}
    \includegraphics[width=.3\linewidth]{imageBlue.png}
    \includegraphics[width=.3\linewidth]{imageGreen.png}
\end{frame}

\section{Calcolo dell'Indice di variabilità}

\begin{frame}
{\Large\textbf{Indici di Variabilità}}
\parbox{\linewidth}{
  \begin{itemize}
    \item  Calcolo DVI: Indice di differenza di vegetazione
    \begin{equation}
        DVI = NIR-red 
    \end{equation}
    \item Calcolo NDVI: Indice di differenza normalizzata delle vegetazioni impiegando i valori delle bande rosse e infrarosse delle immagini
    \begin{equation}
        NDVI = \frac{(NIR-red)}{(NIR+red)}
    \end{equation}
    \item Calcolo NDRE: Normalized Difference Red Edge si basa sulla differenza tra le riflettanze nelle bande del Red Edge e del vicino infrarosso (NIR)
    \begin{equation}
        NDRE = \frac{(NIR-Red Edge)}{(NIR+Red Edge)}
    \end{equation} 
  \end{itemize} 
}
\end{frame}

\section{Calcolo dell'Indice di variabilità}

\begin{frame} 
{\Large\textbf{Indici di Variabilità: DVI}}
\begin{columns}
    \column{0.3\textwidth}
\begin{itemize}
\item Calculate dvi for 2018


I risultati del 2018 sono: min -0.20, max 0.69
\item Calculate dvi for 2022


I risultati del 2022 sono: min -0.19, max 0.67
\end{itemize}
\column{0.5\textwidth}
    \includegraphics[width=.9\linewidth]{DVI.png}
    \label{Fig: DVI_2018-2022}
\end{columns}
\end{frame}

\section{Calcolo dell'Indice di variabilità}

\begin{frame} 
{\Large\textbf{Indici di Variabilità: NDVI}}
\begin{columns}
    \column{0.3\textwidth}
\begin{itemize}
\item Calculate ndvi for 2018


I risultati del 2018 sono: min -0.45, max 0.91
\item Calculate ndvi for 2022


I risultati del 2022 sono: min -0.42, max 0.99
\end{itemize}
\column{0.5\textwidth}
    \includegraphics[width=.9\linewidth]{NDVI.png}
    \label{Fig: NDVI_2018-2022}
\end{columns}    
\end{frame}

\section{Calcolo dell'Indice di variabilità}

\begin{frame} 
{\Large\textbf{Indici di Variabilità: NDRE}}
\begin{columns}
    \column{0.3\textwidth}
\begin{itemize}
\item Calculate ndre for 2018


I risultati del 2018 sono: min -0.57, max 0.72
\item Calculate ndre for 2022


I risultati del 2022 sono: min -0.50, max 0.86
\end{itemize}
\column{0.5\textwidth}
    \includegraphics[width=.9\linewidth]{NDRE.png}
    \label{Fig: NDRE_2018-2022}
\end{columns}  
\end{frame}

\section{Classificazione delle immagini}

\begin{frame}
{\Large\textbf{Classificazione delle immagini e calcolo della frequenza}}
\vspace{-1cm}
Si classificano le immagini impiegando la funzione "im.classify()" e successivamente si svolge il calcolo della relativa frequenza, proporzione e percentuale del numero dei pixel.


\vspace{+0.5cm}
\raggedright\textbf{Percentages Anno 2018}
\begin{itemize} 
\item soil = 23
\item forest = 76
\item glacier = 1.4
\end{itemize} 
\textbf{Percentages Anno 2022}
\begin{itemize} 
\item soil = 17
\item forest = 81
\item glacier = 2.3
\end{itemize} 
\end{frame}

\section{Ggplot e Dataframe}

\begin{frame}
{\Large\textbf{Ggplot e Dataframe}}

Costruisco un dataframe con al suo interno le classi e gli oggetti "y2018", "y2022", con al loro interno le percentuali della classificazione calcolate in precedenza.
I dati sono poi impiegati per creare due grafici a barre per confrontare le distribuzioni delle classi delle immagini a disposizione.


    \centering
    \includegraphics[width=.4\linewidth]{Grafico_2018.png}
    \includegraphics[width=.4\linewidth]{Grafico_2022.png}
\end{frame}
 
\section{Conclusioni}

\begin{frame}
{\Large\textbf{Conclusioni}}

I valori degli indici di vegetazione DVI, NDVI e NDRE analizzati per i due anni 2018 e 2022 mostrano una situazione in generale stabile.
Nonostante la presenza di potenziali patologie vegetali, l'analisi studio indica che tra il 2018 e il 2022 si è verificato un lieve miglioramento della vegetazione sia in termini di quantità (copertura) che qualità (salute fisiologica).
\end{frame}

\end{document}
